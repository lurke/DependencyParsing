\documentclass[12pt,fleqn]{article}

\usepackage[T1]{fontenc}
\usepackage[utf8]{inputenc}
\usepackage[margin=1in]{geometry}
\usepackage{amsmath,amssymb,amsfonts}
\usepackage{graphicx}
\usepackage{enumitem}

\title{CS 187 - Dependency Parsing\\Phase 4: Final Paper Draft}
\author{Lauren Urke, Nathaniel Herman, Henrik Sigstad, Ariel Camperi}
\date{}

\begin{document}
    \maketitle
    \hrule
\begin{abstract}
The abstract ...
\end{abstract}
\section{Introduction}
\begin{itemize}
\item Motivation
\item What we did and why we think it is important
\item Statement of key result
\end{itemize}

\section{Statement of the problem}
\begin{itemize}
\item Further discussion about why we care...
\end{itemize}

\section{Description of the method}
\subsection{Description of data}
To implement Dependency Parsing, we used the Penn Treebank Project Data. This is a data set containing sentences parsed into trees with part of speech tags. 

\subsection{Description of SVMs}
\subsection{Description of dependency parsing}
\subsection{Description of code}

\section{Results}
\subsection{Results}

    \begin{center}
        \begin{tabular}{|c|c|c|}
            \hline Dependency Accuracy & Root Accuracy & Complete Rate \\ \hline
            0.89135 & 0.93675 & 0.32711 \\ \hline
        \end{tabular}
    \end{center}

    \begin{center}
        \begin{tabular}{|c|c|c|}
            \hline Dependency Accuracy & Root Accuracy & Complete Rate \\ \hline
            0.88680 & 0.93894 & 0.32285 \\ \hline
        \end{tabular}
    \end{center}


    \begin{center}
        \begin{tabular}{|l|cccc|cccc|}
            \cline{2-9} \multicolumn{1}{c|}{} & (2,2) & (2,3) & (2,4) & (2,5) & (3,2) & (3,3) & (3,4) & (3,5) \\ \hline
            Dep. Acc. & 0.891 & 0.890 & 0.890 & 0.889 & 0.887 & 0.887 & 0.887 & 0.887 \\
            Root Acc. & 0.937 & 0.928 & 0.934 & 0.930 & 0.932 & 0.929 & 0.928 & 0.927 \\
            Comp. Rate & 0.327 & 0.330 & 0.325 & 0.326 & 0.312 & 0.318 & 0.316 & 0.317 \\ \hline
        \end{tabular}
    \end{center}


\section{Conclusion}
\begin{itemize}
\item Summarize how well we were able to replicate the paper
\item Ideas for improving their method?
\end{itemize}


\end{document}

%%% Local Variables:
%%% mode: latex
%%% TeX-master: t
%%% End:
